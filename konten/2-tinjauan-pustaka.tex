\chapter{TINJAUAN PUSTAKA}

% Ubah konten-konten berikut sesuai dengan isi dari tinjauan pustaka
\section{Hasil penelitian/perancangan terdahulu}
Makalah "\emph{Experimental Evaluation of Internet of Things in the Educational Environment}" oleh Amr Elsaadany dan Mohamed Soliman mempelajari potensi manfaat dan dampak Internet of Things (IoT) di lingkungan pendidikan. Penulis berpendapat bahwa IoT memiliki potensi untuk merevolusi pendidikan dengan memberikan siswa pengalaman belajar yang dipersonalisasi, meningkatkan kolaborasi dan komunikasi, dan membuat pembelajaran lebih menarik dan interaktif.
Penulis melakukan evaluasi eksperimental IoT di lingkungan universitas. Mereka menggunakan berbagai perangkat IoT, termasuk sensor, aktuator, dan tag RFID, untuk mengumpulkan data tentang perilaku dan kinerja siswa. Data tersebut digunakan untuk mengembangkan model pembelajaran yang dipersonalisasi untuk setiap siswa. Model ini kemudian digunakan untuk memberi siswa umpan balik dan saran yang disesuaikan untuk perbaikan.

Rahman et al. (2020) dalam makalahnya menyajikan model pemanfaatan ruang untuk institusi pendidikan tinggi (PT). Model dikembangkan dengan wawancara dan diskusi kelompok terarah dengan pemangku kepentingan PT. Model ini didasarkan pada model system development lifecycle (SDLC) dan menggunakan diagram alir data untuk membuat prototipe model.
Model ini terdiri dari empat komponen utama:
\begin{itemize}
    \item Input: Komponen ini meliputi data fisik ruangan PT, seperti jumlah ruangan, ukuran ruangan, dan tipe ruangan.
    \item Proses: Komponen ini mencakup proses pengumpulan data pemanfaatan ruang, analisis data, dan pengembangan rekomendasi untuk peningkatan pemanfaatan ruang.
    \item Keluaran: Komponen ini mencakup laporan pemanfaatan ruang dan rekomendasi untuk meningkatkan pemanfaatan ruang.
    \item Umpan balik: Komponen ini mencakup umpan balik dari pemangku kepentingan LPT tentang model dan rekomendasinya.
\end{itemize}
Model tersebut diuji di salah satu perguruan tiinggi di Malaysia dan hasilnya menunjukkan bahwa model tersebut mampu meningkatkan pemanfaatan ruang. Model tersebut adalah alat yang berharga untuk HEI yang ingin meningkatkan pemanfaatan ruang dan mengurangi biaya.

Saffari et al. (2021) mengusulkan sistem deteksi okupansi kamera bebas baterai yang menggunakan kamera beresolusi rendah, modul komunikasi backscatter, dan Raspberry Pi 4 Model B. Kamera memanen energi dari cahaya sekitar dan mentransmisikan data ke Raspberry Pi menggunakan komunikasi backscatter. Raspberry Pi menjalankan model pembelajaran mendalam untuk mendeteksi keberadaan manusia.

\section{Teori/Konsep Dasar}

\subsection{Deteksi Objek}
Object detection merupakan bagian dari visi komputer yang berperan untuk mendeteksi objek visual dari kelas tertentu (misalnya, manusia, hewan, mobil, atau bangunan) dalam gambar digital seperti bingkai foto atau video [4]. Teknologi ini banyak digunakan dalam tugas yang berkaitan dengan pengenalan citra, seperti pengenalan wajah, pengenalan aktivitas, penghitungan kendaraan, dan lainnya. Ini juga digunakan untuk melacak objek, misalnya melacak bola selama pertandingan sepak bola, melacak pergerakan tongkat kriket, atau melacak seseorang dalam video [5]. Tujuan dari deteksi objek adalah untuk mengembangkan model komputasi yang menyediakan informasi paling mendasar yang dibutuhkan oleh aplikasi visi komputer, yaitu untuk mengetahui objek apa yang berada di mana.

\subsection{\emph{Internet of Things}}
Internet of things (IoT) menggambarkan objek fisik (atau kelompok objek semacam itu) dengan sensor, kemampuan pemrosesan, perangkat lunak, dan teknologi lain yang menghubungkan dan bertukar data dengan perangkat dan sistem lain melalui Internet atau jaringan komunikasi lainnya [6]. Salah satu kegunaan yang didapat dengan menerapkan IoT adalah perangkat bisa dipantau atau dikontrol selama masih terhubung dalam satu jaringan. Dengan menghubungkan peralatan kedalam jaringan, peralatan ini dimungkinkan untuk melakukan pertukaran data.

\subsection{Okupansi}
Dirujuk dari Kamus Besar Bahasa Indonesia (KBBI), okupansi bermakna hunian. Suatu tempat dikatakan memiliki okupansi jika tempat itu dihuni orang. Dalam bidang properti, okupansi adalah kepadatan suatu ruang atau bangunan [7]. Istilah okupansi ini biasa ditemukan dalam banyak bangunan, dengan komponen berupa luas ruang, kapasitas orang, dan kebutuhan tiap individu. Okupansi tiap ruangan akan berbeda bergantung pada komponen diartas, contoh kecilnya adalah okupansi ruang kerja bisa berbeda dengan okupansi ruang kamar tidur. 

\subsection{\emph{Deep Learning}}
Deep Learning merupakan bagian dari machine learning. Deep learning memungkinkan model komputasi yang terdiri dari beberapa lapisan pemrosesan untuk mempelajari representasi data dengan berbagai tingkat abstraksi [8]. Metode-metode ini telah secara dramatis meningkatkan state-of-the-art dalam pengenalan suara, pengenalan objek visual, deteksi objek dan banyak domain lainnya seperti penemuan obat dan genomik. Berbagai jenis algoritma yang menerapkan deep learning antara lain Convolutional Neural Network (CNN), Recurrent Neural Network (RNN), Long Short Term Memory (LTSM), Self Organizing Maps (SOM), dan lain sebagainya. Semua algoritma ini dalam praktiknya digunakan untuk memproses data-data.

