\chapter*{ABSTRACT}
\begin{center}
  \large
  \textbf{\emph{ANTI-GRAVITY} BASED ENERGY CALCULATION ON OUTER SPACE ROCKETS}
\end{center}
% Menyembunyikan nomor halaman
\thispagestyle{empty}

\begin{flushleft}
  \setlength{\tabcolsep}{0pt}
  \bfseries
  \begin{tabular}{lc@{\hspace{6pt}}l}
  Student Name / NRP&: &Ja'far Shadiq / 07211940000023\\
  Department&: &Computer engineering FTEIC - ITS\\
  Advisor&: &1. Dion Hayu Fandiantoro, S.T., M.T.\\
  & & 2. Eko Pramunanto, S.T., M.T.\\
  \end{tabular}
  \vspace{4ex}
\end{flushleft}
\textbf{Abstract}

% Isi Abstrak
Information technology that exists today is growing rapidly and can offer increased efficiency in various fields. Within the scope of smart building, possible efficiency improvements include ease of management, cost savings, increased environmental sustainability, and others. This study aims to find out how to detect occupancy and to develop a tool that can detect space availability based on occupancy using a camera. This system will be equipped with deep learning technology so that the camera can detect whether or not there are people in the room. This system is expected to be installed in lecture buildings or lecturer offices. This system has the potential to improve the optimization of effective use of space, cost savings, increased environmental sustainability, and others.

\vspace{2ex}
\noindent
\textbf{Keywords: \emph{Internet of Things, Occupancy, Smart Building}}