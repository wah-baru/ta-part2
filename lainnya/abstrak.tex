\chapter*{ABSTRAK}
\begin{center}
  \large
  \textbf{Sistem Pemantauan Ketersediaan Ruangan pada Gedung Perkuliahan Berbasis Okupansi Menggunakan Kamera}
\end{center}
\addcontentsline{toc}{chapter}{ABSTRAK}
% Menyembunyikan nomor halaman
\thispagestyle{empty}

\begin{flushleft}
  \setlength{\tabcolsep}{0pt}
  \bfseries
  \begin{tabular}{ll@{\hspace{6pt}}l}
  Nama Mahasiswa / NRP&:& Ja'far Shadiq / 07211940000023\\
  Departemen&:& Teknik Komputer FTEIC - ITS\\
  Dosen Pembimbing&:& 1. Dion Hayu Fandiantoro, S.T., M.T..\\
  & & 2. Eko Pramunanto, S.T., M.T.\\
  \end{tabular}
  \vspace{4ex}
\end{flushleft}
\textbf{Abstrak}

% Isi Abstrak
Teknologi informasi yang ada saat ini berkembang pesat, dan dapat menawarkan peningkatan efisiensi di berbagai bidang. Dalam lingkup smart building, peningkatan efisiensi yang memungkinkan diantaranya kemudahan manajemen, penghematan biaya, peningkatan kelestarian lingkungan, dan lainnya. Penelitian ini bertujuan untuk mengetahui cara mendeteksi okupansi, dan mengembangkan alat yang dapat mendeteksi ketersediaan ruang berdasarkan okupansi menggunakan kamera. Sistem ini akan dilengkapi dengan teknologi deep learning, sehingga kamera dapat mendeteksi ada tidaknya orang dalam ruangan. Sistem ini diharapkan dapat dipasang di gedung perkuliahan atau kantor dosen. Sistem ini berpotensi untuk meningkatkan optimalisasi penggunaan ruang yang efektif, penghematan biaya, peningkatan kelestarian lingkungan, dan lainnya.

\vspace{2ex}
\noindent
\textbf{Kata Kunci: \emph{Internet of Things, Okupansi, Smart Building}}